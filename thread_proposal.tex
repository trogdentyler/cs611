\documentclass[12pt]{article}

\usepackage{geometry}

\title{CS 611: Story Proposal}
\author{Tyler Trogden}
\date{Due: 02/03/2023}

\begin{document}
\maketitle

    \section*{}
        Topological data mining (TDM) refers to the process of extracting
        information from data sets using methods from the field of topology.
        In particular, the subfields of abstract and differential topology 
        have given scientists many tools for analyzing data sets in novel ways.
        In recent years, TDM has grown in popularity due to the availability of 
        large data sets and the development of new algorithms, software, and
        hardware. This increased interest and arsenal of tools has
        led to the creation of many new applications of TDM, which include
        applications in the fields of biology, chemistry, physics, and
        artificial intelligence, to name a few.

        With this surge in literature my goal is to tell the story of the
        current state of TDM and how it arrived at where it is. This primarily
        begins with the inception of abstract topology and its advancement to 
        topological data analysis (TDA). I will focus on the applications of 
        TDA to various fields, and will discuss the current state of the 
        field in terms of algorithms and experimental results. I will also 
        discuss the challenges that TDA faces in the future, and will 
        discuss possible solutions to these challenges.

        The motivation behind studying TDA is primarily due to its applicability
        to a wide variety of fields, however, I am particularly interested in its
        application to deep learning as a novel method for analyzing DNNs themselves.
        Novel research in this area has the potential to lead to new insights
        into the inner workings of DNNs, which could lead to new ways to
        improve their performance. One particular tool from TDA that I am
        interested in is persistent homology (PH), which is a method for analyzing
        the topology of data sets. In recent years PH has been used to analyze
        the topology of DNNs, and I am interested in learning more about this
        application and what it can reveal.

\end{document}